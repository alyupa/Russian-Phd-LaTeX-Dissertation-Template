\chapter*{Заключение}                       % Заголовок
\addcontentsline{toc}{chapter}{Заключение}  % Добавляем его в оглавление

%% Согласно ГОСТ Р 7.0.11-2011:
%% 5.3.3 В заключении диссертации излагают итоги выполненного исследования, рекомендации, перспективы дальнейшей разработки темы.
%% 9.2.3 В заключении автореферата диссертации излагают итоги данного исследования, рекомендации и перспективы дальнейшей разработки темы.
%% Поэтому имеет смысл сделать эту часть общей и загрузить из одного файла в автореферат и в диссертацию:

Основные результаты работы заключаются в следующем:
\begin{enumerate}
 \item Выведены гиперболизированные системы уравнений фильтрации двумя способами;
 \item Разработаны алгоритмы решения полученных систем уравнений;
 \item Аппроксимированы уравнения явными разностыми схемами;
 \item Написаны программы для проведения вычислений, ориентированные на использование суперкомпьютеров;
 \item Проведены вычислительные эксперименты;
 \item Проведены сравнения решений известных тестовых задач о просачивании многофазных жидкостей в пористых средах под действием силы тяжести или заданного градиента давления с IMPES-методом и решениями других авторов.
\end{enumerate}

Результаты тестовых расчетов показали, что использование предложенной модификацированной модели и явной позволяет существенно увеличить шаг по времени как минимум на порядок, а значит ускорить численное решение. Кроме того, явные схемы позволяют более эффективно применять параллельные вычисления.
При этом качественный характер течений не меняется, а погрешность вычислений остается приемлемой.

Данная тема в дальнейшем может быть разработана путем учета ряда дополнительных физических явлений, а также обобщения на случай течений многокомпонентных многофазных жидкостей в пористых средах.
