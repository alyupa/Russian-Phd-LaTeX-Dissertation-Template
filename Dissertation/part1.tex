\chapter{Математические модели многофазного течения в пористой среде} \label{ch:ch1}

\newcommand*{\pd}[3][]{\ensuremath{\dfrac{\partial^{#1} #2}{\partial #3^{#1}}}}
\newcommand*{\dd}[3][]{\ensuremath{\dfrac{\mathrm d^{#1} #2}{\mathrm d #3^{#1}}}}
\newcommand{\grad}{\mathop{\mathrm{grad}}\nolimits}
\newcommand{\diver}{\mathop{\mathrm{div}}\nolimits}

В работе проводится моделирование двухфазной и трехфазной фильтрации
несмешивающихся жидкостей и газа с учетом их сжимаемости. Скелет породы считаем неподвижным, пористость и плотность породы -- постоянными во всем рассматриваемом объеме, среду -- изотропной, жидкие фазы -- слабосжимаемыми, газ -- идеальным.

\section{Классическая модель фильтрации} \label{sec:ch1/sec1}

Приведем классическую модель. Система уравнений для описания двухфазной изотермической фильтрации в пористой среде
может быть представлена в виде~\cite{Aziz-Settari},~\cite{Basniev}:
\begin{subequations} \label{eq:classic_system}
  \begin{align}[left = \empheqlbrace\,]
    &\pd {(\phi \rho_i S_i)}{t} + \diver{(\rho_i \overrightarrow{u_i})} = q_i, \label{eq:classic_system1} \\
    &\overrightarrow{u_i} = -\frac{K k_i}{\mu_i}(\grad {P_i} - {\rho}_i g \grad {z}), \label{eq:classic_system2} \\
    &P_n = P_w + P_{cnw}({S}_w), \label{eq:classic_system3} \\
    &\sum_{i}S_i = 1, \label{eq:classic_system4} \\
    &k_i = k_i({S}_i), \label{eq:classic_system5} \\
    &\rho_i = \rho_i(P_i), \label{eq:classic_system6} \\
    &i = w,n \nonumber .
  \end{align}
\end{subequations}

Здесь
$K$ -- абсолютная проницаемость среды,
$\phi$ -- пористость среды,
$g$ -- ускорение свободного падения,
$S_i$ -- насыщенность $i$-й фазы,
$P_i$ -- давление $i$-й фазы,
$P_{cnw}$ -- капиллярное давление на границе раздела двух фаз,
${\rho}_i$ -- плотность $i$-й фазы,
$q_i$ -- источник $i$-й фазы,
$\overrightarrow{u_i}$ -- скорость фильтрации $i$-й фазы,
$\mu_i$ -- вязкость $i$-й фазы,
$k_i$ -- относительная фазовая проницаемость $i$-й фазы,

\eqref{eq:classic_system1} -- уравнения неразрывности,
\eqref{eq:classic_system2} -- закон Дарси,
\eqref{eq:classic_system3} -- соотношение между давлениями фаз,
\eqref{eq:classic_system4} -- свойство суммы насыщенностей фаз по определению,
\eqref{eq:classic_system5} -- функции относительных фазовых проницаемостей,
\eqref{eq:classic_system6} -- уравнения состояния.

Относительные фазовые проницаемости задаются в~соответствии с~
приближением Стоуна\cite{Aziz-Settari}.
Капиллярные давления описываются согласно приближению в модели Паркера\cite{Parker}.

В модели трехфазной фильтрации $i = w,n,g$, система дополняется уравнением $P_g = P_n + P_{cgn}({S}_g)$.

\section{Гиперболическая КГД модель фильтрации} \label{sec:ch1/sec2}

При построении гиперболической модели вместо \eqref{eq:classic_system1} строится модифицированное уравнение неразрывности по аналогии с~квазигазодинамической системой уравнений~\cite{Chetverushkin-Mathmod}):
\begin{equation}
 \label{mass_mod}
  \frac{\partial (\phi \rho_i S_i)}{\partial t} + \tau \frac{\partial^2 (\rho_i S_i)}{\partial t^2}
  \diver(\rho_i \overrightarrow{u_i}) = q_i + l c_i \cdot \diver(\grad(\rho_i S_i)),
\end{equation}
где $l$ -- характерный масштаб (расстояние порядка сотни размеров зерен породы~\cite{Chetverushkin}),
$c_i$ -- скорость распространения звука в $i$-ой среде, $\tau$ -- характерное время (время установления внутреннего равновесия в объеме с характерным размером $l$).

В~данной работе предложена модель трехфазной фильтрации, основанная на кинетическом подходе, по аналогии с тем, как это было сделано для одно- и двухфазной фильтрации в работах~\cite{Mathmod-2010,Mathmod-2011}.

\section{Гиперболическая модель фильтрации, учитывающая релаксацию потока массы} \label{sec:ch1/sec3}

Перейдем к построению гиперболической модели, учитывающей релаксацию потока массы. 
Запишем уравнение~\eqref{eq:classic_system1} кратко в виде
\begin{equation} \label{eq:short_system}
   \pd{(\phi \rho_i S_i)}{t} + \diver{ \overrightarrow{Q_i}} = q_i, i = w,n,
\end{equation}
где $\overrightarrow{Q_i}$ -- поток массы фазы.
Пусть $\overrightarrow{Q_i^D} = \rho_i \overrightarrow{u_i}$ -- поток Дарси.
В модели~\eqref{eq:classic_system} $\overrightarrow{Q_i} = \overrightarrow{Q_i^D}$.
Рассмотрим релаксацию потока:
\begin{equation}
 \overrightarrow{Q_i} = \overrightarrow{Q_i^D} - \tau \pd{\overrightarrow{Q_i}}{t},
\end{equation}
где $\tau$ -- параметр релаксации, довольно малая величина, характеризующая время установления равновесия в системе. Откуда следует, что
\begin{equation} \label{eq:div_relax}
 \diver{\overrightarrow{Q_i}} = \diver{\overrightarrow{Q_i^D}} - \tau \diver{\pd{\overrightarrow{Q_i}}{t}}.
\end{equation}

Продифференцируем уравнения~\eqref{eq:short_system} по времени, домножим на $\tau$:
\begin{equation}
  \tau \pd[2]{(\phi \rho_i S_i)}{t} + \tau \pd{(\diver{ \overrightarrow{Q_i}})}{t} = \tau \pd{q_i}{t}.
\end{equation}

С учетом~\eqref{eq:div_relax} и ~\eqref{eq:short_system} получим модифицированное уравнение неразрывности путем введения релаксации потока массы:

\begin{equation} \label{eq:mod_system}
  \tau \pd[2]{(\phi \rho_i S_i)}{t} + \pd{(\phi \rho_i S_i)}{t} + \diver{ \overrightarrow{Q_i^D}} = q_i + \tau \pd{q_i}{t}.
\end{equation}
