\chapter{Алгоритмы решения задач фильтрации} \label{ch:ch2}

\section{Алгоритм расчета классической модели фильтрации} \label{sec:ch2/sec1}

Найти точное общее решение системы уравнений ~\eqref{eq:classic_system} не представляется возможным в силу свойств системы, поэтому используем решение, полученное с помощью распространенного и проверенного на практике IMPES-метода с достаточно малым расчетным шагом по времени.
В этом случае имеем неявное разностное уравнение для давления (матрица коэффициентов трехдиагональная) и явное для насыщенности. Полученная СЛАУ решается итерационным численным методом.

\section{Алгоритм расчета КГД модели фильтрации} \label{sec:ch2/sec2}

Численное решение полученной системы уравнений разбито на~этапы. После
применения начальных условий на каждом шаге по времени выполняется следующая
последовательность действий: 
\begin{enumerate} 
\item Применение граничных условий.
\item Вычисление давлений $P_n$,
$P_g$ через $P_w$ и капиллярные давления. 
\item Вычисление плотностей фаз. 
\item
Нахождение относительных фазовых проницаемостей фаз и~вязкости.
\item Определение
коэффициентов в~законе Дарси для~фаз. 
\item 
\label{roS} 
Определение ${\rho}_iS_i$ на~
следующем временном шаге из~уравнения~\eqref{mass_mod}.
\item 
\label
{Newton} Решение нелинейной системы из~пяти уравнений методом Ньютона,
в результате чего находим $P_w$, $S_w$, $S_n$, $S_g$ на следующем шаге по~времени.
\item Сохранение полученных значений переменных в~текстовый файл
в~формате, подходящем для~визуализации.
\item Обмены данными при~многопроцессорных вычислениях.
\end{enumerate}
Для вычислений на шаге~\ref{roS} выбран класс явных двухслойных схем на равномерных декартовых сетках,
допускающих эффективное распараллеливание решения.
Рассмотрены две различные схемы:
\begin{itemize}
\item с направленными разностями;
\item с центральными разностями.
\end{itemize}

Рассмотрим подробнее шаг~\ref{Newton}. В каждом расчетном узле возникает 
нелинейная система уравнений.
Находим решение методом Ньютона~\cite{Kalitkin} за семь
итераций. Каждая итерация 
представляет собой следующую последовательность действий:
\begin{eqnarray*}
  \begin{aligned}
    F_1=\ &\rho_w(P_w, T) S_w - (\widehat{\rho_w S_w}) \\
    F_2=\ &\rho_n(P_w+P_{cnw}(S_w)) S_n - (\widehat{\rho_n S_n}) \\
    F_3=\ &\rho_g(P_w+P_{cnw}(S_w)+P_{cgn}(S_g)) S_g - (\widehat{\rho_g S_g}) \\
    F_4=\ & m \Big{(}S_w \big{(}\rho_w(P_w, T) H_w(T) - P_w\big{)} \\
	 &+ S_n \big{(}\rho_n(P_w+P_{cnw}(S_w), T) H_n(T) - P_w - P_{cnw}(S_w)\big{)} \\
	 &+ S_g \big{(}\rho_g(P_w+P_{cnw}(S_w)+P_{cgn}(S_g), T) H_g(T) - P_w - P_{cnw}(S_w) - P_{cgn}(S_g)\big{)}
	 \Big{)}\\
	 &+ (1-m) (\rho_r H_r - P_w) - \widehat{E} \\
    F_5=\ &S_w+S_n+S_g - 1
  \end{aligned}
\end{eqnarray*}
\begin{equation}
A=
\begin{pmatrix}
\dfrac{\partial{F_1}}{\partial{P_w}} & \dfrac{\partial{F_1}}{\partial{S_w}} & \dfrac{\partial{F_1}}{\partial{S_n}} & \dfrac{\partial{F_1}}{\partial{S_g}} & \dfrac{\partial{F_1}}{\partial{T}}\\[3mm]
\dfrac{\partial{F_2}}{\partial{P_w}} & \dfrac{\partial{F_2}}{\partial{S_w}} & \dfrac{\partial{F_2}}{\partial{S_n}} & \dfrac{\partial{F_2}}{\partial{S_g}} & \dfrac{\partial{F_2}}{\partial{T}}\\[3mm]
\dfrac{\partial{F_3}}{\partial{P_w}} & \dfrac{\partial{F_3}}{\partial{S_w}} & \dfrac{\partial{F_3}}{\partial{S_n}} & \dfrac{\partial{F_3}}{\partial{S_g}} & \dfrac{\partial{F_3}}{\partial{T}}\\[3mm]
\dfrac{\partial{F_4}}{\partial{P_w}} & \dfrac{\partial{F_4}}{\partial{S_w}} & \dfrac{\partial{F_4}}{\partial{S_n}} & \dfrac{\partial{F_4}}{\partial{S_g}} & \dfrac{\partial{F_4}}{\partial{T}}\\[3mm]
\dfrac{\partial{F_5}}{\partial{P_w}} & \dfrac{\partial{F_5}}{\partial{S_w}} & \dfrac{\partial{F_5}}{\partial{S_n}} & \dfrac{\partial{F_5}}{\partial{S_g}} & \dfrac{\partial{F_5}}{\partial{T}}\\[3mm]
\end{pmatrix}
\end{equation}

Тогда

\begin{equation}
\begin{pmatrix}
P_w\\
S_w\\
S_n\\
S_g\\
T
\end{pmatrix}^{new}
=
\begin{pmatrix}
P_w\\
S_w\\
S_n\\
S_g\\
T
\end{pmatrix}
-A^{-1}
\begin{pmatrix}
F_1\\
F_2\\
F_3\\
F_4\\
F_5
\end{pmatrix}
\end{equation}\\

Обращаем матрицу $A$ методом Гаусса с~выбором главного элемента\cite{Kalitkin}.

\section{Алгоритм расчета модели фильтрации с релаксацией потока массы} \label{sec:ch2/sec3}

Рассмотрим двухфазную модели фильтрации~\eqref{eq:mod_system}.

Пусть 
\begin{subequations} \label{eq:subs}
  \begin{align}
    &q_t = \dfrac{q_n}{\rho_n} + \dfrac{q_w}{\rho_w}, \\
    &P_{avg} = \dfrac{P_n + P_w}{2}, \\
    &P_n = P_{avg} + \dfrac{P_c}{2},
     P_w = P_{avg} - \dfrac{P_c}{2}, \\
    &c_n = \dfrac{1}{\rho_n} \dd{\rho_n}{P_n},
     c_w = \dfrac{1}{\rho_w} \dd{\rho_w}{P_w}, \\
    &\lambda_n = -\dfrac{K k_n}{\mu_n},
     \lambda_w = -\dfrac{K k_w}{\mu_w}, \\
    & f_w = \dfrac{\lambda_w}{\lambda_n + \lambda_w},
      h_w = - \dfrac{\lambda_n\lambda_w}{\lambda_n + \lambda_w} \dd{P_c}{S_w}, \\
    & \overrightarrow{u_t} = \overrightarrow{u_n} + \overrightarrow{u_w}.
  \end{align}
\end{subequations}

Вспомогательные выкладки:
\begin{equation} \label{eq:calcs1}
 \begin{aligned}
  \pd[2]{(\phi \rho_i S_i)}{t} &= \rho_i S_i \pd[2]{\phi}{t} + \phi S_i \pd[2]{\rho_i}{t} + \phi \rho_i \pd[2]{S_i}{t} + \\
  &+ 2S_i \pd{\phi}{t} \cdot \pd{\rho_i}{t} + 2\rho_i \pd{\phi}{t} \cdot \pd{S_i}{t} + 2\phi \pd{\rho_i}{t} \cdot \pd{S_i}{t} = \\
  &= \rho_i S_i \left(  \dd[2]{\phi}{P_{avg}} \cdot {\left( \pd{P_{avg}}{t}\right) }^2 + \dd{\phi}{P_{avg}} \cdot \pd[2]{P_{avg}}{t} \right) +\\
  &+ \phi S_i \left( \dd[2]{\rho_i}{P_i} \cdot {\left( \pd{P_i}{t}\right) }^2 + \dd{\rho_i}{P_i} \cdot \pd[2]{P_i}{t} \right) + \phi \rho_i \pd[2]{S_i}{t} +\\
  &+ 2S_i \dd{\phi}{P_{avg}} \cdot \pd{P_{avg}}{t} \cdot \dd{\rho_i}{P_i} \cdot \pd{P_i}{t} + 2\rho_i \dd{\phi}{P_{avg}} \cdot \pd{P_{avg}}{t} \cdot \pd{S_i}{t} +\\
  &+ 2\phi \dd{\rho_i}{P_i} \cdot \pd{P_i}{t} \cdot \pd{S_i}{t},
 \end{aligned}
\end{equation}
\begin{equation} \label{eq:calcs2}
 \begin{aligned}
  \pd[2]{\phi}{t} &= \dd[2]{\phi}{P_{avg}} \cdot {\left( \pd{P_{avg}}{t}\right) }^2 + \dd{\phi}{P_{avg}} \cdot \pd[2]{P_{avg}}{t}, \\
  \pd[2]{\rho_i}{t} &= \dd[2]{\rho_i}{P_i} \cdot {\left( \pd{P_i}{t}\right) }^2 + \dd{\rho_i}{P_i} \cdot \pd[2]{P_i}{t}, \\
  \pd{q_i}{t} &= \dd{q_i}{P_{avg}} \cdot \pd{P_{avg}}{t},
 \end{aligned}
\end{equation}

Путем преобразований получим уравнения для нахождения
среднего давления и водонасыщенности:
\begin{equation} \label{eq:full_pressure}
 \begin{gathered}
   \frac{1}{\rho_n} \diver{ (\rho_n\lambda_n\grad{ P_{avg}})} + \frac{1}{\rho_w} \diver{ (\rho_w\lambda_w\grad{ P_{avg}})}  +\\
   + \frac{1}{2} \left[ \frac{1}{\rho_n} \diver{ (\rho_n\lambda_n\grad{ P_c})} + \frac{1}{\rho_w} \diver{ (\rho_w\lambda_w\grad{ P_c})} \right]  +\\
   + q_t = \left[ \dd{\phi}{P_{avg}} + \phi (S_n c_n + S_w c_w) \right] \left( \pd{P_{avg}}{t} + \tau \pd[2]{P_{avg}}{t} \right) +\\
   + \frac{1}{2} \phi (S_n c_n - S_w c_w) \left( \pd{P_{c}}{t} + \tau \pd[2]{P_c}{t} \right) +\\
   + g \left[ \frac{1}{\rho_n} \diver{ (\rho_n^2\lambda_n\grad{ z})} + \frac{1}{\rho_w} \diver{ (\rho_w^2\lambda_w\grad{ z})} \right] +\\
   + \tau \left[ \dd[2]{\phi}{P_{avg}} + \phi \left( \frac{1}{\rho_n} S_n \dd[2]{\rho_n}{P_n} + \frac{1}{\rho_w} S_w \dd[2]{\rho_w}{P_w} \right) \right. + \\
   \left. + 2(S_n c_n + S_w c_w) \dd{\phi}{P_{avg}} \right] {\left( \pd{P_{avg}}{t}\right) }^2 +\\
   + \tau \left[ \phi \left( \frac{1}{\rho_n} S_n \dd[2]{\rho_n}{P_n} - \frac{1}{\rho_w} S_w \dd[2]{\rho_w}{P_w} \right) \right. +\\
   \left. + (S_n c_n - S_w c_w) \dd{\phi}{P_{avg}} \right] \pd{P_{avg}}{t} \cdot \pd{P_c}{t} +\\
   + \frac{1}{4} \tau \phi \left( \frac{1}{\rho_n} S_n \dd[2]{\rho_n}{P_n} + \frac{1}{\rho_w} S_w \dd[2]{\rho_w}{P_w} \right) {\left( \pd{P_c}{t}\right) }^2 -\\
  - \tau \left(\frac{1}{\rho_n}\dd{q_n}{P_{avg}} + \frac{1}{\rho_w}\dd{q_w}{P_{avg}}\right) \pd{P_{avg}}{t} +\\
   + 2 \tau \phi \left(c_n \pd{S_n}{t} + c_w \pd{S_w}{t} \right) \pd{P_{avg}}{t}+ \\
   + \tau \phi \left(c_n \pd{S_n}{t} - c_w \pd{S_w}{t} \right) \pd{P_c}{t},
 \end{gathered}
\end{equation}
\begin{equation} \label{eq:full_saturation}
 \begin{gathered}
  \diver{(\rho_w h_w \grad{ S_w})} - \diver{(\rho_w f_w[\overrightarrow{u_t} + \lambda_n(\rho_w - \rho_n)g \grad{ z}])} + q_w =\\
  = \rho_i S_i \dd{\phi}{P_{avg}} \cdot \pd{P_{avg}}{t} + \phi S_i \dd{\rho_i}{P_i} \cdot \pd{P_i}{t} + \phi \rho_i \pd{S_i}{t} +\\
  + \tau \pd[2]{(\phi \rho_w S_w)}{t} - \tau \dd{q_w}{P_{avg}} \cdot \pd{P_{avg}}{t}.
 \end{gathered}
\end{equation}

Уравнения~\eqref{eq:full_pressure},~\eqref{eq:full_saturation} решаются с помощью трехслойных явных разностных схем с центральными разностями.
