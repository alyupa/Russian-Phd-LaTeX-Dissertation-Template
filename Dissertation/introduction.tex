\chapter*{Введение}                         % Заголовок
\addcontentsline{toc}{chapter}{Введение}    % Добавляем его в оглавление
\textbf{Актуальность темы исследования}. На протяжении многих лет математическое моделирование течения многофазных жидкостей в пористых средах занимает важное место в решении таких индустриально-технологических задач, как добыча нефти и газа, проектирование гидротехнических сооружений, анализ экологических проблем. 
А высокопроизводительные вычислительные системы открывают для этого новые возможности.

\textbf{Степень разработанности темы исследования}. Обширная литература посвящена описанию процессов фильтрации
жидкости в пористых средах. Это и фундаментальные теоретические труды, и описания инженерных практик, и пособия для студентов технических вузов. Как правило, в рассматриваемых задачах имеются отличительные особенности,
которые делают универсальный подход к моделированию этих процессов невозможным.
Параметры моделей существенно зависят от свойств флюидов и среды.
В результате чего в настоящее время существует большое разнообразие моделей фильтрации в пористых средах.

\textbf{Цели и задачи}:
\begin{itemize}
 \item Вывести гиперболизированные системы уравнений фильтрации;
 \item Аппроксимировать уравнения явными разностыми схемами;
 \item Разработать алгоритм решения полученных систем уравнений;
 \item Разработать комплекс программ для проведения вычислений, ориентированную на использование суперкомпьютеров;
 \item Провести вычислительные эксперименты;
 \item Проверить модель на тестовых задачах.
\end{itemize}

\textbf{Научная новизна}. Разработана уникальная модель многофазных течений слабосжимаемых жидкостей в пористых средах.
Для моделирования слабосжимаемых течений жидкости в пористой среде подход, учитывающий релаксацию потоков, применяется впервые.
Также представлен новый алгоритм расчетов, адаптированный для использования высокопроизводительных вычислительных систем.

\textbf{Теоретическая и практическая значимость работы}. Гиперболизация систем уравнений активно применяется в гидродинамике и газодинамике, но ее применение  до сих пор мало исследовано в области моделирования многофазных течений в пористых средах. В данной работе показана эффективность применения этого подхода для моделирования фильтрации слабосжимаемых жидкостей. Проведены сравнения решений известных тестовых задач о просачивании многофазных жидкостей в пористых средах под действием силы тяжести или заданного градиента давления с IMPES-методом и решениями других авторов, что показало возможность дальнейшего развития данного направления.

\textbf{Методология и методы исследования}. Проведен обзор литературы по теме исследования. Исследованы модели многофазной фильтрации, описанные другими авторами и применяемые на практике.
Для проведения расчетов по классической модели и ее модификациям написаны программы на языках C++ и Python с использованием научных библиотек NumPy, SciPy. Для проведения параллельных расчетов на суперкомпьютерах использованы технологии MPI, CUDA.
Параметры модели, обеспечивающие устойчивость расчетов, подбирались эмпирически.
Результаты расчетов были визуализированы с помощью свободного программного обеспечения.

\textbf{Положения, выносимые на публичное представление}:
\begin{itemize}
 \item Разработаны две гиперболизированные системы уравнений для моделирования процессов многофазной фильтрации: основанная на квазигазодинамической системе уравнений, а также основанная на методе релаксации потоков. Предложены алгоритмы явного типа для их численной реализации;
 \item Гиперболизация уравнений не оказала существенного влияния на точность полученных результатов;
 \item Предложенные алгоритмы решения полученных систем уравнений позволяют существенно увеличить шаг по времени при использовании явных разностных схем.
\end{itemize}

\textbf{Степень достоверности и апробация результатов}. Результаты исследований опубликованы в 13 публикациях в рецензируемых изданиях
~\citeauthor{matmod2014, matmod2015, preprint12016, preprint22016, preprint12018, preprint22018, simul2014, modelling2016, problems2018, proc13th, proc19th, proc2017, proc2018}. Также основные результаты были изложены и обсуждались на 4 международных конференциях:
\begin{enumerate}
 \item Thirteenth International Seminar <<Mathematical models and modeling in laser-plasma processes and advanced science technologies>> (LPPM3-2015), Petrovac, Montenegro, May 30-June 6, 2015~\citeauthor{proc13th};
 \item 19th International Conference on Circuits, Systems, Communications and Computers (CSCC 2015), Zakynthos Island, Greece, July 16-20, 2015~\citeauthor{proc19th};
 \item International Conference <<Mathematical Modeling and Computational Physics, 2017>> (MMCP2017), Dubna, July 3-7, 2017~\citeauthor{proc2017};
 \item 6th European Conference on Computational Mechanics (ECCM 6) 7th European Conference on Computational Fluid Dynamics (ECFD 7), Glasgow, UK, 11-15 June 2018~\citeauthor{proc2018}.
\end{enumerate}
 
