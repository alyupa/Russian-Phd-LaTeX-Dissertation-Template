\chapter*{Введение}                         % Заголовок
\addcontentsline{toc}{chapter}{Введение}    % Добавляем его в оглавление
\textbf{Актуальность темы исследования}. На протяжении многих лет математическое моделирование фильтрации занимает
очень важное место при разработке технологий добычи нефти и газа, при прогнозировании показателей разработки месторождений, 
при проектировании, постройке и эксплуатации гидротехнических и мелиоративных сооружений, в горном деле,
в решении экологических проблем. Это позволяет снизить накладные расходы и добиться большей эффективности при планировании работ.
А высокопроизводительные вычислительные системы открывают новые возможности для решения этих индустриально-технологических задач.

\textbf{Степень разработанности темы исследования}. Выведены гиперболизированные системы уравнений фильтрации двумя способами.
Первая модель разработана на основе принципа минимальных размеров и техники дифференциальных приближений.
Вторая модель получена с помощью введения релаксации потока массы в уравнение неразрывности.
Предложены алгоритмы, использующие аппроксимацию модифицированных уравнений трехслойными явными
разностными схемами. Проведен ряд вычислительных экспериментов.

\textbf{Цели и задачи}:
\begin{enumerate}
 \item Вывести гиперболизированные системы уравнений фильтрации;
 \item Разработать алгоритм решения полученных систем уравнений;
 \item Аппроксимировать уравнения явными разностыми схемами;
 \item Написать программу для проведения вычислений, ориентированную на использование суперкомпьютеров;
 \item Провести вычислительные эксперименты;
 \item Проверить модель на тестовых задачах.
\end{enumerate}

\textbf{Научная новизна}. Разработана уникальная модель многофазных течений слабосжимаемых жидкостей в пористых средах.
Также представлен новый алгоритм расчетов, адаптированный для использования высокопроизводительных вычислительных систем.

\textbf{Теоретическая и практическая значимость работы}. Гиперболизация систем уравнений активно применяется в гидродинамике и газодинамике, но ее применение  до сих пор мало исследовано в области моделирования многофазных течений в пористых средах. В данной работе показана эффективность применения этого подхода для моделирования фильтрации слабосжимаемых жидкостей. Проведены сравнения решений известных тестовых задач о просачивании многофазных жидкостей в пористых средах под действием силы тяжести или заданного градиента давления с IMPES-методом и решениями других авторов, что показало возможность дальнейшего развития данного направления.

\textbf{Методология и методы исследования}. Для проведения расчетов по классической модели и ее модификациям написаны программы на языках C++ и Python с использованием научных библиотек NumPy, SciPy. Для проведения параллельных расчетов на суперкомпьютерах использованы технологии MPI, CUDA.
Параметры модели, обеспечивающие устойчивость расчетов, подбирались эмпирически в зависимости от размера шага по времени.
Результаты расчетов были визуализированы с помощью свободного программного обеспечения.

\textbf{Положения, выносимые на публичное представление}:
\begin{enumerate}
 \item Гиперболизация систем уравнений не меняет качественного характера многофазных течений слабосжимаемых жидкостей в пористом подземном
пространстве;
 \item Предложенные алгоритмы решения полученных систем уравнений позволяют существенно снизить ограничение на шаг по времени при использовании явных разностных схем.
\end{enumerate}

\textbf{Степень достоверности и апробация результатов}. Результаты исследований опубликованы в 6 статьях в рецензируемых журналах
~\cite{matmod2014, matmod2015, preprint12016, preprint22016, preprint12018, preprint22018}. Также основные результаты были изложены и обсуждались на 4 российских и международных конференциях~\cite{proc13th, proc19th, proc2017, proc2018}:
\begin{enumerate}
 \item Thirteenth International Seminar <<Mathematical models and modeling in laser-plasma processes and advanced science technologies>> (LPPM3-2015), Petrovac, Montenegro, May 30-June 6, 2015;
 \item 19th International Conference on Circuits, Systems, Communications and Computers (CSCC 2015), Zakynthos Island, Greece, July 16-20, 2015;
 \item International Conference <<Mathematical Modeling and Computational Physics, 2017>> (MMCP2017), Dubna, July 3-7, 2017;
 \item 6th European Conference on Computational Mechanics (ECCM 6) 7th European Conference on Computational Fluid Dynamics (ECFD 7), Glasgow, UK, 11-15 June 2018.
\end{enumerate}
 
